\begin{figure}[H]
    \centering
    \pgfmathdeclarefunction{gauss}{2}{\pgfmathparse{1/(#2*sqrt(2*pi))*exp(-((x-#1)^2)/(2*#2^2))}}
    \begin{tikzpicture}[
        declare function={
            g(\x)=1/(1*sqrt(2*pi))*exp(-((\x-0)^2)/(2*1^2));
        }
    ]
        \begin{axis}[
            samples=300,
            axis line style={draw=none},
            ymax=0.45,
            xmax=5.5,
            ticks=none,
            height=7cm,
            width=12cm,
            enlargelimits=false,
            clip=false,
            axis on top,
        ]
            % 局部染色
            \path[fill=red!30] (axis cs: 1.645,0) -- plot[domain=1.645:3] (axis cs: {\x}, {g(\x)}) -- (axis cs: 3,0) -- cycle;
            \path[fill=forestgreen!30] (axis cs: -3,0) -- plot[domain=-3:1.645] (axis cs: {\x}, {g(\x)}) -- (axis cs: 1.645,0) -- cycle;
            % 概率密度曲线
            \addplot [domain=-3:3, line width=1pt] {gauss(0,1)};
            \draw (axis cs: -3.5,0) -- (axis cs: 3.5,0);
            % 辅助性的虚线与坐标点
            \draw[dashed] (axis cs: 1.645,-0.01) -- (axis cs: 1.645,0.4);
            \draw (axis cs: 0,-0.01) -- (axis cs: 0,0.01);
            \node[below] at(axis cs: 0,-0.01) {$\theta_0$}
                node[below] at(axis cs: 1.645,-0.01) {$c$}
                node at(axis cs: -1.1,0.3) {$F_0$};
        \end{axis}
    \end{tikzpicture}
\end{figure}